\documentclass{article}

\usepackage[a4paper, top=0.5in, bottom=0.5in, left=0.5in, right=0.5in]{geometry}


\usepackage{tabularx}
\usepackage[hidelinks]{hyperref}

%\usepackage{doublespace}
%\setstretch{1.2}

\usepackage{ae}
\usepackage[utf8]{inputenc}
%\usepackage[T1]{fontenc}
\usepackage{CV}

\begin{document}

\pagestyle{empty}

%Ueberschrift
\begin{center}
\makebox[\textwidth][s]{\noindent\Large{\textsc{Bart{\l}omiej Mariusz Skorulski} $\bullet$
\textsc{Academic Curriculum Vitae}}}
\end{center}
\vspace{-0.7cm}
\section{Contact Details}

\begin{flushleft}
Phone: +34-626803069\\
Email: \href{mailto:bartekskorulski@gmail.com}{bartekskorulski@gmail.com}
\end{flushleft}
% \section{Personal Details}
% \begin{flushleft}
%  Date of birth: 1 August, 1974 \\
%  Nationality: Polish\\
% % Place of birth: Bialystok, Poland \\
% % Present Citizenship: Polish\\
% \end{flushleft}

\section{Professional Experience}

\begin{CV}

\item[2019--present] {\bf Senior Data Scientist \it Research Team, Alpha (Telefonica), Barcelona, Spain}

  Working on digital phenotyping though data and machine/deep learning.
  
\item[2006--2013] {\bf Associate Professor \it Universidad Cat\'olica del Norte,
    Antofagasta, Chile}
  
  Taught graduate courses: Hyperbolic Dynamic; Ergodic Theory; Topics
  on Complex Analysis; Functional Analysis.  Taught undergraduate
  courses: Complex Analysis I and II; Real Analysis; Calculus II;
  Functional Analysis.

  Thesis Supervision: Irene Inoquio (PhD. 2010), Felipe Correa
  (M.Sc. 2012), Sebastian Sarmiento (M.SC. 2010), Adriana Tapia
  (M.Sc. 2008), Mery Choque (M.Sc. 2009) Oscar Santamaria (M.Sc.,
  2007), Francisco Bravo (Bachelor, 2006).

\item[Jan--Aug 2009, Feb--May 2007, Aug--Nov 2003 and more]
  {\ \\ \bf Visiting Professor \it University of North Texas, Denton, USA}
  
  Research, taught undergraduate courses \emph{Probability Models}.

\item[2005--2006] {\bf Postdoctoral Position \it Universidad Cat\'olica del Norte}

  Taught graduate course on Ergodic Theory.
  
\item[2018--2019] {\bf Staff Insight Analyst \it Schibsted, Barcelona, Spain}
  
  Schibsted is an international media group that is one of the world’s leading online classified ads
  businesses (leboncoin, subito, finn, blocket, fotocasa, yappo, etc). I mainly work with Messaging
  Team helping them with: A/B tests and data analysis, NLP models, data pipelines and dashboards.  I
  also work with Experimentation Team helping them with statistical evaluation of AB Tests and with
  Performance Dashboard Team providing them expertise on tracking, data pipelines and Machine
  Learning.

  I also mentor new Product Analysis and provide training on Data Access and Machine Learning for
  other teams.
  
\item[2017--2018] {\bf Senior Data Scientist \it SCRM (Lidl), Barcelona, Spain}

  SCRM is a company that has created Lidl Plus program (Lidl's version of loyalty program). I was
  leading a team that has created and put to production: recommend er system, forecasting tool,
  that is predicting demand.  We also were setting up hypothesis-driven decision-making environment
  and has been developing A/B Test Tool.

\item[2015--2017] {\bf Data Scientist \it King (Activision/Blizzard), Barcelona, Spain}

  I was helping to launch probably the best bubble shooter ever Bubble Witch 3 and working on
  improving the experience of 300 milions players of King's games. I was translating business needs
  to technical requirements and AB tests, and then working with development teams to ensure its
  correct implementation and tracking. I was also developing an analysis strategy and performing
  analysis of complex scenarios, for example, helping introducing soft currency and assuring its
  correct balance.

\item[Jun 2017, Jun 2018] {\ \\ \bf Lecturer \it Universitat Politècnica de Catalunya, Barcelona, Spain}

  Thought "Big Data Management with R" course on MESIO UPC-UB Summer Schools.

\item[May--Aug 2015] {\bf Volunteer \it Oxfam Intermón, Unidad de
    Monitoreo y evaluación de Campañas, Barcelona, Spain}

  Creating web application that analyzes and visualize Oxfam Intermón
  campaigns on Twitter. It accesses twitter API and stores relevant
  information in a data base. It provides different
  statistics together with different
  visualizations like graph of connections, maps showing
  where participants come from etc. It exports data to csv, gephi and
  produces periodic reports in pdf.  
  
\item[Jan- Apr 2015] {\bf SEM Technologist \it Clacktion
    Barcelona, Spain}

  Creating Search Engine Marketing (SEM) campaigns, developing
  software for reporting, automatic control and creation of large
  online marketing campaigns.
  

\item[2013-2014] {\bf Mathematical modelling \it Soluciones de Gestión y Apoyo a Empresas S.L,
    Zaragoza, Spain}

  Mathematical modelling in electric engineering, developing
  applications,  writing and typesetting for engineering journal
  ``Soluçoes''.


\item[Sep--Nov 2010] {\bf Visiting Professor \it Universit{\'e} Lille 1, France}
  
\item[Jun--Aug 2007] {\bf Visiting Professor \it Polish Academy of Sciences, Warsaw, Poland}

\item[Apr--Jul 2003] {\bf Marie Curie Scholarship \it University of Warwick, Coventry, UK}
    
\item[Mar--May 2002] {\bf Research School \it Centro di Ricerca Mathematica, Scuola Normale
    Superiore, Pisa, Italy.}


\item[2004--2005] {\bf Assistant Professor \it Institute of Mathematics of the Polish Academy of
    Sciences, Warsaw, Poland}

  Research position.
  
\item[2000--2004] {\bf PhD student \it Warsaw University of Technology, Warsaw, Poland}

  Taught undergraduate courses: Complex Analiysis (Faculty of
  Mathematics and Information Science), Mathematics I, Mathematics III
  (Faculty of Mechatronics), Mathematics I (Faculty of Production Engineering)

\item[1998--2000] {\bf Assistant Professor \it Faculty of Mathematics and Physics, University
    of Bia{\l}ystok}

  Taught graduate courses: Functional Analysis I and II, Orders and
  Numbers.  Taught undergraduate courses: Topology, Combinatorics.

\end{CV}



\section{Higher Education}
\begin{CV}
\item[2000--2005] \emph{Ph.D. in Mathematics}, Faculty of Mathematics and
  Information Sciences, Warsaw University of Technology.

  Thesis: \emph{Metric Properties of the Julia set of some meromorphic
    functions}; Supervisor: Janina Kotus.

\item[1997--1999] \emph{M.Sc. (Hons) in Mathematics}, Faculty of Mathematics
  and Physics, University of Bialystok.

\item[1994--1997] {\em Bachelor's Degree (Hons) in Mathematics,} Faculty of
  Mathematics and Physics, Warsaw University.

\end{CV}


\section{List of Publications}
\begin{enumerate}
\item Book: \emph{Distance Expanding Random Maps, Thermodynamical
    Formalism, Gibbs Measures and Fractal Geometry}, Lecture Notes in
  Mathematics,  Springer, 2011. (with V. Mayer and M. Urba{\'n}ski)
\item Finer Fractal Geometry for Analytic Families of Conformal
  Dynamical Systems, (with M. Urbanski), \emph{Dynamical Systems 29
    (2014), 369--398.}
\item Regularity and Irregularity of Fiber Dimensions of
  Non-Autonomous Dynamical Systems, (with V. Mayer and M. Urbanski),
  {\em Annales Academiae Scientiarum Fennicae Mathematica 38 (2013), 489--514.}.
\item Dynamical Rigidity of Transcendental Meromorphic Functions,
  (with M. Urbanski), {\em Nonlinearity 25 (8)} (2012), 2337--2348.
\item The Law of Iterated Logarithm and Equilibrium Measures Versus
  Hausdorff Measures for Dynamically semi-Regular Meromorphic
  Functions, (with M. Urbanski), to apper {\em "Further Developments in
  Fractals and Related Fields", in "Trends in Mathematics" of
  Birkhauser.}
\item Thermodynamic formalism of transcendental entire maps of finite
  type. {\em Monatshefte f\"ur Mathematik 152}, 2 (2007), 105--123.
  (with Ion Coiculescu)
\item Perturbations in the {S}peiser class.  {\em Rocky Mountain
    Journal of Mathematics 37}, 3 (2007), 763--800. (with Ion Coiculescu)
\item Multifractal analysis for the exponential family.  {\em Discrete
    Contin. Dyn. Syst. 16}, 4 (2006), 857--869.  (with Godofredo Iommi)
\item The existence of conformal measures for some transcendental
  meromorphic functions. In {\em Complex Dynamics: Twenty-Fife Years
    after the Appearance of the Mandelbrot Set}, vol.~396 of {\em
    Contemp. Math.} Amer. Math. Soc., Providence, RI, 2006,
  pp.~169--201.
\item Metric properties of the {J}ulia set of some meromorphic
  functions with an asymptotic value eventually mapped onto a pole.
  {\em Math. Proc. Cambridge Philos. Soc. 139}, 1 (2005), 117--138.
\item Non-ergodic maps in the tangent family. {\em Indag. Math. (N.S.)
    14}, 1 (2003), 103--118.\\
  % \end{enumerate}
  \ \\
  \
{\sc Publications related to the project Mizar }
%\begin{enumerate}
\item First-countable, sequential, and {F}rechet spaces.  {\em
    Formalized Mathematics 7}, {\bf 1} (1998), 81--86.
\item The sequential closure operator in sequential and {F}rechet spaces.
{\em Formalized Mathematics 8}, {\bf 1} (1999), 47--54.
\item Lim-inf convergence.  {\em Formalized Mathematics 9}, {\bf 2}
  (2001), 237--240.
\item The {T}ichonov {T}heorem. {\em Formalized Mathematics 9}, {\bf 2} (2001), 373--376.
\end{enumerate}                 



\section{Research Grants}

\begin{CV}
\item[2006--2009] Chilean Science Foundation (FONDECYT) 

  Project: Invariant Measures and Thermodynamic Formalism for
  Meromorphic Functions (Fondecyt N. 11060538), Principal Researcher.
\item[2006--2009] Chilean Science Foundation (CONICYT) 

  Low Dimensional Dynamical Systems Network, Young Researcher.
\item[2006--2009] European Union

  Marie Curie Research Training Network CODY: Conformal Structures and
  Dynamics, Associate Researcher of Warsaw node.
\item[2005] Warsaw University of Technology

  Geometric Rigidity and Multifractal Analysis for Meromorphic
  Function.
\item[2004] Warsaw University of Technology

  Project: Investigation of Properties of the Julia Set of Functions
  of the Form $f=R(\exp(z))$, where $R$ is a Rational Map, Principal
  Researcher.
\item[2004] Warsaw University of Technology

  Project: Hausdorff Dimension and Geometry of the Julia Set for Meromorphic
  Functions.
\item[2003--2006] Polish Science Foundation (KBN)

  Project: Conformal Dynamical Systems and Geometry of the Fractal Sets.
\item[2002--2003] Warsaw University of Technology 

  project: Conformal and Invariant Measures and Hausdorff Dimension of the
  Julia Set for Transcendental Meromorphic Functions.
\item[2001] Warsaw University of Technology

  Project: Invariant Measures for Meromorphic Functions.
\item[2000--2004] European Union

  Marie Curie Research Training Network CALCULEMUS: Systems for
  Integrated Computation and Deduction.
\item[1997--1998] Polish Science Foundation (KBN)

  Project: Natural Artificial Intelligence Research in Automated
  Reasoning.
\end{CV}

% \section{Given Talks}

% \begin{CV}

% \item[Jul 2013] ``Regularity and irregularity of fiber dimension of non--auto\-nomous dynamical systems'', 

%   Thermodynamic Formalism and applications, %July 8-12, 2013
%   PUC, Santiago, Chile

% \item[Jan 2012] ``Hausdorff Dimension of radial Julia sets of meromorphic functions'',

%   Mini-Workshop: Thermodynamic Formalism, Geometry and Stochastics,
%   Mathematisches Forschungsinstitut Oberwolfach, Germany

% %\item {\em Workshop on Holomorphic Dynamics} University of Warwick,
% %  UK, 6.12--11.12.2004
% %\item {\em Summer School and Conference on Dynamical Systems} The
% %  Abdus Salam International Centre for Theoretical Physics, Trieste,
% %  Italy, 19.07--6.08.2004,

% \item[Dec 2009] ``J-stability of Random Perturbation of Meromorphic
%   Mappings'',

%   Dynamics and Analysis Seminar, University of North Texas, Denton TX,
%   USA.

% \item[Sep 2009] ``Distance Expanding Random Mappings, Thermodynamic
%   Formalism, Gibbs Measures and Fractal Geometry'', 

%   {\em Dynamical
%     Systems at Valparaíso, a satellite conference of the III CLAM},
%   Valparaiso, Chile.

% \item[Aug 2009] ``Conjuntos Aleatorios'',

%   \emph{El Congreso de Matemática Capricornio, COMCA 2009},
%   Anto\-fagasta, Chile.

% \item[Dec 2008] Minicourse: ``Cómo medir fractales'', 

%   {\em Escuela de Sistemas Dinámicos}, La Paz, Bolivia.


% \item[Oct 2008] ``Random Fractal Geometry'', 

%   Dynamics and Analysis Seminar, University of North Texas, Denton TX,
%   USA.

% \item[Aug 2008] ``Exponential Misiurewicz does not give acip''

% Conference: {\em COMCA 2008}, Iquique, Chile.

% \item[Aug 2007] ``Dynamics of some meromorphic functions with an
%   asymptotic value eventually mapped onto infinity''

%   Conference: {\em First Joint International Meeting between the AMS
%     and the PTM}, Warsaw, Poland.

% \item[Apr 2008] Geometría de conjuntos aleatorios,

%   Seminario de Sistemas Dinámicos, Pontificia Universidad Cat\'olica
%   de Chile.

% \item[Jun 2007] ``Dynamics of some meromorphic functions''

%   Conference: {\em Conformal Structures and Dynamics. The current
%     state-of-art and perspectives}, University of Warwick, Coventry,
%   UK.

% %%%%%%%%%%%%%%%%%%%%%%%%

% \item[Nov 2006] ``Dinámica de funciones meromorfas''

%   Conference: XII Congreso Boliviano de Matemática, Oruro, Bolivia.
% \item[Oct 2006] ``Thermodynamic formalism for some meromorphic functions''

%   Conference: {\em Coloquio LV de sistemas dinamicos}, Depto Matematicas,
%   Universidad de Chile, Santiago, Chile.
% \item[Jul 2006] ``Dynamics of real tangent family'' 

%   Conference: {\em Dynamical Days}, Pontificia Universidad Católica de
%   Chile, Santiago, Chile.
% \item[Mar 2006] Minicourse: ``Introducci{\'o}n a Systemas
%   Dyn{\'a}micos''

%   Conference: {\em Aniversary of the Mathematical Department},
%   Universidad Mayor de San Andr\'es, La Paz, Bolivia.
% \item[Feb 2006] Birkhoff Ergodic Theorem

%   Universidad Mayor de San Andr\'es, La Paz, Bolivia.
% \item[Nov 2005] ``Multifractal analysis for the exponential family'' 

%   Conference: {\em LIV Coloquio de Sistemas Din\'amicos,}, Pontificia
%   Universidad Católica de Chile, Santiago, Chile, 7--8.11.2005.
% \item[Oct 2005] ``Multifractal analysis for the exponential family''

%   Univesidad T\'ecnica Federico Santa Mar\'ia, Valparaiso, Chile.
% \item[Aug 2005] ``Multifractal analysis for the exponential family''

%   Conference: {\em IV Workshop on Dynamical Systems}, San Pedro de
%   Atacama, Chile.
% \item[May 2005] Minicourse: ``Zbiory Julii dla rodziny kradratowej''

%   {Spring School on Dynamical Systems}, Bedlewo, Poland.
% \item[Apr 2005] ``W{\l}asno{\'s}ci metryczne zbi{\'o}row Julii pewnych funkcji
%   meromorficznych'' 
  
%   Uniwersytet Jagiellonski, Krakow, Poland.
% \item[Dec 2004] ``Lematy o po{\l}aczeniu i zamykaniu'' 
  
%   Warsaw University, Poland.
%   % {\tt https://www.mimuw.edu.pl/badania/seminaria/zud/}
% % \item {\em Hiperboliczne funkcje wykladnicze, miary Gibbsa i 
% %       nieograniczone spektrum multifraktalne }, Instytut Matematyczny 
% %     Polskiej Akademii Nauk, Poland, 15.11.2005.
% \item[Jun 2004] ``Iteration of meromorphic functions''

%   Conference: {\em Complex Dynamics: Twenty Five Years After the
%     Appearance of the Mandelbrot Set,} Snowbird, USA.
% \item[May 2004] Minicourse: Dynamika funkcji zespolonych,

%   {Spring School on Dynamical Systems}, Bedlewo, Poland.


% \item[Nov 2003] Metric properties of some family of transcendental
%   meromorphic functions

%   University Of North Texas, Denton, USA.
% %  \item {\em Wieze i zbieznosc renormalizacji, part I}, Instytut Matematyczny 
% %    Polskiej Akademii Nauk, Poland, 21.02.2005.
% %   \item {\em Renormalizacje}, Instytut Matematyczny Polskiej Akademii Nauk, 
% %     Poland, 24.01.2005.
% %   \item {\em Lematy o po{\l}aczeniu i zamykaniu, part II}, Warsaw University, 
% %     Poland, 07.01.2005.
%     % {\tt https://www.mimuw.edu.pl/badania/seminaria/zud/}
% % \item {\em On Thermodynamic Formalism of hyperbolic entire maps}, 
% %   Instytut Matematyczny Polskiej Akademii Nauk, Poland, 23.03.2004.
% %   \item {\em Metric properties of some meromorphic functions with 
% %       an asymptotic value eventually mapped to a pole},
% %     Instytut Matematyczny Polskiej Akademii Nauk, Poland, 06.01.2004.
% %   \item {\em Meromorphic dynamics}, The University of Warwick, Coventry, UK
% %     April 2003.
% %   \item {\em The finer geometry and dynamics of exponential family},
% %     The University of Warwick, Coventry, UK April 2003.
% \item[Jun 2003] `Metric properties of some family of
%   transcendental meromorphic functions' 

%   Conference: {\em Geometric Aspects of Dynamical Systems}, University
%   of Warwick, Coventry, UK.
% %\item {\em Symbolic Dynamics and
% %    Ergodic Theory}, University of Warwick, Coventry, UK,
% %  7.07-18.07.2003, 
%   \item[Apr 2003] `Non-ergodic maps in tangent family'

%     Conference: {\em Holomorphic Dynamics}, University of Warwick,
%     Coventry, UK.
% %\item {\em Holomorphic Iteration,
% %    Non-Uniform Hyperbolicity}, Stefan Banach Internarional
% %  Mathematical Center, Warszawa, 22.05-25.05.2002.
% % \item {\em Szko"la z U"lad"ow Dynamicznych},
% %   B"edlewo, 28.04--1.05.2005,
% % \item {\em Szko"la z U"lad"ow Dynamicznych},
% %   B"edlewo, 7.05--9.05.2004,
% % \item {\em Warsztaty
% %     z Teorii Aproksymacji}, Uniwersytet Jagielo"nski, Krak"ow,
% %   23.09-28.09.2002, 
% %\item {\em Geometric, probabilistic and
% %    variational methods in variational methods in dynamical systems
% %    II}, PhD Winter School, SISSA -- International School for Advanced
% %  Study, Trieste, W"lochy, 11.02 -- 01.03.2002, 
% %\item {\em Geometric, probilistic and variational methods in
% %    va\-ria\-tio\-nal methods in dynamical systems I}, PhD Winter
% %  School, SISSA -- International School for Advanced Study, Trieste,
% %  W"lochy, 15.01-25.01.2002, 
% %\item {\em Summer School on Dynamical
% %    Systems}, University of G\"ottingen, Niem\-cy, 10.07-20.07.2001.
% %\item {\em
% %    Iteracje funkcji zespolonych i przekszta"lce"n odcinka},
% %  seminarium prowadzone przez Prof. Feliksa Przytyckiego, IM PAN, od 2001 roku,
% %\item {\em Układy Dynamiczne}, MIM UW, od 2001 roku.
% %\item {\em Fraktale}, seminarium prowadzone przez dr hab. Ann"e Zdunik i dr.  
% %Krzysztofa Bara"nskiego, MIM UW, 2000-2003.  
% \end{CV}
    
% \section{Other scientific activities}

% \begin{CV}
% \item[May 2010] Member of Organising Committee of the Conference
%   \emph{ Bicentennial Workshop on Dynamical Systems}, San Pedro de
%   Atacama, Chile.

% \item[Aug 2009] Member of Organising Committee of the Conference
%   \emph{El Congreso de Matemática Capricornio, COMCA}, Antofagasta, Chile.

% \item[Aug 2009] Coordinator of the Session of Dynamical Systems of the
%   Conference \emph{El Congreso de Matemática Capricornio, COMCA},
%   Antofagasta, Chile.

% \item[May 2009] Member of Organising Committee of the Conference
%   \emph{Dynamical Systems II, Denton}, Denton TX, USA.


% \item[Dec 2007] Member of Organising Committee of the Workshop
%   \emph{Dynamical Systems Days}, Antofagasta, Chile

% \item[2006--2013] Organising of the Seminar \emph{DEA (Dinámica,
%     Ecuaciones Y Acciones)}, Universidad Cat\'olica del Norte,
%   Antofagasta, Chile.

% \end{CV}

% \section{Given Talks}

% \begin{CV}
% \item 

% %\item {\em Workshop on Holomorphic Dynamics} University of Warwick,
% %  UK, 6.12--11.12.2004
% %\item {\em Summer School and Conference on Dynamical Systems} The
% %  Abdus Salam International Centre for Theoretical Physics, Trieste,
% %  Italy, 19.07--6.08.2004,
% \item[Aug 2007] ``Dynamics of some meromorphic functions with an
%   asymptotic value eventually mapped onto infinity''

%   Conference: {\em First Joint International Meeting between the AMS
%     and the PTM}, Warsaw, Poland.

% \item[Jun 2007] ``Dynamics of some meromorphic functions''

%   Conference: {\em Conformal Structures and Dynamics. The current
%     state-of-art and perspectives}, University of Warwick, Coventry,
%   UK.

% %%%%%%%%%%%%%%%%%%%%%%%%

% \item[Nov 2006] ``Dinámica de funciones meromorfas''

%   Conference: XII Congreso Boliviano de Matemática, Oruro, Bolivia.
% \item[Oct 2006] ``Thermodynamic formalism for some meromorphic functions''

%   Conference: {\em Coloquio LV de sistemas dinamicos}, Depto Matematicas,
%   Universidad de Chile, Santiago, Chile.
% \item[Jul 2006] ``Dynamics of real tangent family'' 

%   Conference: {\em Dynamical Days}, Pontificia Universidad Católica de
%   Chile, Santiago, Chile.
% \item[Mar 2006] Minicourse: ``Introducci{\'o}n a Systemas
%   Dyn{\'a}micos''

%   Conference: {\em Aniversary of the Mathematical Department},
%   Universidad Mayor de San Andr\'es, La Paz, Bolivia.
% \item[Feb 2006] Birkhoff Ergodic Theorem

%   Universidad Mayor de San Andr\'es, La Paz, Bolivia.
% \item[Nov 2005] ``Multifractal analysis for the exponential family'' 

%   Conference: {\em LIV Coloquio de Sistemas Din\'amicos,}, Pontificia
%   Universidad Católica de Chile, Santiago, Chile, 7--8.11.2005.
% \item[Oct 2005] ``Multifractal analysis for the exponential family''

%   Univesidad T\'ecnica Federico Santa Mar\'ia, Valparaiso, Chile.
% \item[Aug 2005] ``Multifractal analysis for the exponential family''

%   Conference: {\em IV Workshop on Dynamical Systems}, San Pedro de
%   Atacama, Chile.
% \item[May 2005] Minicourse: ``Zbiory Julii dla rodziny kradratowej''

%   {Spring School on Dynamical Systems}, Bedlewo, Poland.
% \item[Apr 2005] ``W{\l}asno{\'s}ci metryczne zbi{\'o}row Julii pewnych funkcji
%   meromorficznych'' 
  
%   Uniwersytet Jagiellonski, Krakow, Poland.
% \item[Dec 2004] ``Lematy o po{\l}aczeniu i zamykaniu'' 
  
%   Warsaw University, Poland.
%   % {\tt https://www.mimuw.edu.pl/badania/seminaria/zud/}
% % \item {\em Hiperboliczne funkcje wykladnicze, miary Gibbsa i 
% %       nieograniczone spektrum multifraktalne }, Instytut Matematyczny 
% %     Polskiej Akademii Nauk, Poland, 15.11.2005.
% \item[Jun 2004] ``Iteration of meromorphic functions''

%   Conference: {\em Complex Dynamics: Twenty Five Years After the
%     Appearance of the Mandelbrot Set,} Snowbird, USA.
% \item[May 2004] Minicourse: Dynamika funkcji zespolonych,

%   {Spring School on Dynamical Systems}, Bedlewo, Poland.


% \item[Nov 2003] Metric properties of some family of transcendental
%   meromorphic functions

%   University Of North Texas, Denton, USA.
% %  \item {\em Wieze i zbieznosc renormalizacji, part I}, Instytut Matematyczny 
% %    Polskiej Akademii Nauk, Poland, 21.02.2005.
% %   \item {\em Renormalizacje}, Instytut Matematyczny Polskiej Akademii Nauk, 
% %     Poland, 24.01.2005.
% %   \item {\em Lematy o po{\l}aczeniu i zamykaniu, part II}, Warsaw University, 
% %     Poland, 07.01.2005.
%     % {\tt https://www.mimuw.edu.pl/badania/seminaria/zud/}
% % \item {\em On Thermodynamic Formalism of hyperbolic entire maps}, 
% %   Instytut Matematyczny Polskiej Akademii Nauk, Poland, 23.03.2004.
% %   \item {\em Metric properties of some meromorphic functions with 
% %       an asymptotic value eventually mapped to a pole},
% %     Instytut Matematyczny Polskiej Akademii Nauk, Poland, 06.01.2004.
% %   \item {\em Meromorphic dynamics}, The University of Warwick, Coventry, UK
% %     April 2003.
% %   \item {\em The finer geometry and dynamics of exponential family},
% %     The University of Warwick, Coventry, UK April 2003.
% \item[Jun 2003] `Metric properties of some family of
%   transcendental meromorphic functions' 

%   Conference: {\em Geometric Aspects of Dynamical Systems}, University
%   of Warwick, Coventry, UK.
% %\item {\em Symbolic Dynamics and
% %    Ergodic Theory}, University of Warwick, Coventry, UK,
% %  7.07-18.07.2003, 
%   \item[Apr 2003] `Non-ergodic maps in tangent family'

%     Conference: {\em Holomorphic Dynamics}, University of Warwick,
%     Coventry, UK.
%\item {\em Holomorphic Iteration,
%    Non-Uniform Hyperbolicity}, Stefan Banach Internarional
%  Mathematical Center, Warszawa, 22.05-25.05.2002.
% \item {\em Szko"la z U"lad"ow Dynamicznych},
%   B"edlewo, 28.04--1.05.2005,
% \item {\em Szko"la z U"lad"ow Dynamicznych},
%   B"edlewo, 7.05--9.05.2004,
% \item {\em Warsztaty
%     z Teorii Aproksymacji}, Uniwersytet Jagielo"nski, Krak"ow,
%   23.09-28.09.2002, 
%\item {\em Geometric, probabilistic and
%    variational methods in variational methods in dynamical systems
%    II}, PhD Winter School, SISSA -- International School for Advanced
%  Study, Trieste, W"lochy, 11.02 -- 01.03.2002, 
%\item {\em Geometric, probilistic and variational methods in
%    va\-ria\-tio\-nal methods in dynamical systems I}, PhD Winter
%  School, SISSA -- International School for Advanced Study, Trieste,
%  W"lochy, 15.01-25.01.2002, 
%\item {\em Summer School on Dynamical
%    Systems}, University of G\"ottingen, Niem\-cy, 10.07-20.07.2001.
%\item {\em
%    Iteracje funkcji zespolonych i przekszta"lce"n odcinka},
%  seminarium prowadzone przez Prof. Feliksa Przytyckiego, IM PAN, od 2001 roku,
%\item {\em Układy Dynamiczne}, MIM UW, od 2001 roku.
%\item {\em Fraktale}, seminarium prowadzone przez dr hab. Ann"e Zdunik i dr.  
%Krzysztofa Bara"nskiego, MIM UW, 2000-2003.  
%\end{CV}

% \section{Languages}
% \begin{table}[h] %\centering
% \begin{tabular}{p{2cm}>{\bfseries}p{2.5cm}p{3cm}}
% & Polish  & native \\
% & English  & very good \\
% & Spanish & very good\\
% & Russian & basic
% \end{tabular}
% \end{table}

% \pagebreak

% \section{References}

% \noindent These persons are familiar with my professional qualifications and my character:

% \begin{table}[h]
% \begin{tabular}{@{}lll@{}}
% \textbf{Prof. Dr. Pro. Fessor} \\
% Thesis supervisor & Phone: & +31-50-312.3456\\
% P.O. Box 800 & Fax: & +31-50-567.123\\
% 9700 AV Groningen & Email: & p.fessor@xxx.rug.nl \\
% The Netherlands \\
% \end{tabular}
% \end{table}

% \vspace{2\baselineskip}
% \noindent Groningen, \today



\end{document}

%Tabellen
\begin{table}[htbp] \centering%
\begin{tabular}{lll}\hline\hline
1 & 2 & 3 \\ \hline
1 & \multicolumn{2}{c}{2} \\
\hline
\end{tabular}
\caption{Titel\label{Tabelle: Label}}
\end{table}


                                % 

%%% Local Variables:
%%% mode: latex
%%% TeX-master: t
%%% End:
