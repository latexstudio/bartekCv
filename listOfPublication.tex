\documentclass{article}

\begin{document}




\section*{List of Publications}
\begin{enumerate}
\item Book: \emph{Distance Expanding Random Maps, Thermodynamical
    Formalism, Gibbs Measures and Fractal Geometry}, Lecture Notes in
  Mathematics,  Springer, 2011. (with V. Mayer and M. Urba{\'n}ski)
\item Finer Fractal Geometry for Analytic Families of Conformal
  Dynamical Systems, (with M. Urbanski), \emph{Dynamical Systems 29
    (2014), 369--398.}
\item Regularity and Irregularity of Fiber Dimensions of
  Non-Autonomous Dynamical Systems, (with V. Mayer and M. Urbanski),
  {\em Annales Academiae Scientiarum Fennicae Mathematica 38 (2013),
    489--514.}.
\item The Law of Iterated Logarithm and Equilibrium Measures Versus
  Hausdorff Measures for Dynamically semi-Regular Meromorphic
  Functions, (with M. Urbanski), \emph{Further Developments in
    Fractals, Related Fields, Trends in Mathematics, 213-234,
    Birkhauser, 2013.}
\item Dynamical Rigidity of Transcendental Meromorphic Functions,
  (with M. Urbanski), {\em Nonlinearity 25 (8)} (2012), 2337--2348.
\item Thermodynamic formalism of transcendental entire maps of finite
  type. {\em Monatshefte f\"ur Mathematik 152}, 2 (2007), 105--123.
  (with Ion Coiculescu)
\item Perturbations in the {S}peiser class.  {\em Rocky Mountain
    Journal of Mathematics 37}, 3 (2007), 763--800. (with Ion Coiculescu)
\item Multifractal analysis for the exponential family.  {\em Discrete
    Contin. Dyn. Syst. 16}, 4 (2006), 857--869.  (with Godofredo Iommi)
\item The existence of conformal measures for some transcendental
  meromorphic functions. In {\em Complex Dynamics: Twenty-Fife Years
    after the Appearance of the Mandelbrot Set}, vol.~396 of {\em
    Contemp. Math.} Amer. Math. Soc., Providence, RI, 2006,
  pp.~169--201.
\item Metric properties of the {J}ulia set of some meromorphic
  functions with an asymptotic value eventually mapped onto a pole.
  {\em Math. Proc. Cambridge Philos. Soc. 139}, 1 (2005), 117--138.
\item Non-ergodic maps in the tangent family. {\em Indag. Math. (N.S.)
    14}, 1 (2003), 103--118.
% \end{enumerate}
% {\sc Publications related to the project Mizar }
% \begin{enumerate}
% \item First-countable, sequential, and {F}rechet spaces.  {\em
%     Formalized Mathematics 7}, {\bf 1} (1998), 81--86.
% \item The sequential closure operator in sequential and {F}rechet spaces.
% {\em Formalized Mathematics 8}, {\bf 1} (1999), 47--54.
% \item Lim-inf convergence.  {\em Formalized Mathematics 9}, {\bf 2}
%   (2001), 237--240.
% \item The {T}ichonov {T}heorem. {\em Formalized Mathematics 9}, {\bf 2} (2001), 373--376.
\end{enumerate}

\end{document}

%%% Local Variables:
%%% mode: latex
%%% TeX-master: t
%%% End:
